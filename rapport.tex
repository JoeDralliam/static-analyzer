\documentclass{article}


\usepackage[utf8]{inputenc}
\usepackage[T1]{fontenc}
\usepackage[francais]{babel}
\usepackage{listings}
\author{Jun Maillard} 
\title{Analyseur statique} 
\date{\today}


\begin{document}

\lstset{language=C}

\maketitle

Dans le cadre du cours de \textit{Sémantique et vérification}, un
analyseur statique à base de domaines abstraits a été réalisé.

\section{Domaines}
\subsection{Domaines non relationnels} 

Les domaines de valeurs (domaines des constantes et des intervalles)
implémentent les cinq opérateurs arithmétiques, leurs inverses et les
opérateurs de comparaisons. Seul l'opération inverse de modulo
(\verb|bwd_mod|) est implémentée de manière
imprécise (identité). 

Le domaine des parties maintient une association des variables à des
domaines de valeurs. L'élément $\bot$ est représenté par l'association
vide.

Ces domaines sont respectivement implémentés dans les modules
\verb|Constant|, \verb|Interval| et \verb|DomainOfValueDomain|.

\subsection{Domaines relationnels} 

Les domaines relationnels d'Apron sont supporté à travers le foncteurs
\verb|ApronDomain.Make|. L'intégration a nécessité de modifier
légèrement l'interface de \verb|DOMAIN| en ajoutant en paramètre à la
valeur \verb|bottom| la liste des variables du programmes.

\section{Itérateurs}
\subsection{SCfg : Simplification du graphe de flot de contrôle}

Avant l'analyse du graphe, les appels de fonctions sont supprimé du
graphe de flot de contrôle et remplacé par des sauts inconditionnels (\verb|skip|).
Un arc est aussi rajouté entre le n\oe{}ud de fin d'initialisation et le
n\oe{}ud d'entrée de \verb|main|.

\subsection{Analyse en avant}

L'analyse en avant est implémentée dans le module \verb|Forward| à
l'aide d'une worklist.

Les points d'élargissement sont placés les n\oe{}uds de
destination des arcs correspondant à un saut en arrière dans la
source. Lorsqu'un post-pointfixe est atteint, l'itérateur réalise des
itérations décroissantes. Le nombre maximal de ces itérations est
paramétrable.

Les assertions peuvent être toutes assumées ou toutes ignorées.  Si
elles sont assumées, elles se comportent comme des gardes. Sinon,
elles se comportent comme des skip. Une fois les invariants calculés,
l'analyseur détermine les assertions ayant échoué en comparant, pour chaque assertion, le
domaine résultant de la garde par la négation de l'assertion appliquée au domaine
source à $\bot$.

\subsection{Analyse en arrière}

L'analyse en arrière est implémentée dans le module \verb|Backward|

L'analyse en arrière part d'une assertion ayant échoué lors de
l'analyse en avant et parcours le graphe en sens inverse. Les
domaines initiaux sont ceux résultant de l'analyse en avant, à
l'exception du domaine du n\oe{}ud précédant l'assertion : c'est la fraction ne satisfaisant
pas l'assertion. Les assertions sont ignorées lors de cette analyse.

Une fois les invariants calculés, un parcours de graphe permet de
déterminer la trace d'une exécution où l'assertion n'est pas
satisfaite, ou au contraire de garantir que l'assertion est toujours vérifiée.

\section{Interface}

Le module \verb|Main| gère les options passées en ligne de commande.

\subsection{Options de comportement}

\begin{itemize}
\item \verb|-domain| : spécifie le domaine abstrait à utiliser parmi
  \textbf{constant}, \textbf{interval}, \textbf{octagon},
  \textbf{polyhedra}.
\item \verb|-backward-analysis|, \verb|-no-backward-analysis| :
  active/désactive l'analyse en arrière.
\item \verb|-skip-assert|, \verb|-no-skip-assert| : ignore/assume les
  assertions.
\item \verb|-max-decreasing-iterations| : spécifie le nombre
  d'itérations décroissantes lors du calcul d'un point fixe.
\end{itemize}

\subsection{Options d'affichage}
\begin{itemize}
\item \verb|-false-alarms|, \verb|-no-false-alarms| : spécifie
  l'affichage des fausses alarmes.
\item \verb|-invariants|, \verb|-no-invariants| : spécifie l'affichage
  des invariants.
\item \verb|-graph|, \verb|-no-graph| : spécifie l'affichage du graphe
  de flots de contrôle.
\end{itemize}

\section{Mise en application}

Dans cette section, on examine quelques exemples de programmes
illustrant les différentes fonctionnalités de l'analyseur.

\subsection{Élargissement et itérations décroissantes}

\begin{lstlisting}[frame=single]
  int x = rand(-100, 24), y;
  int main()
  {
    while(x <= 50)
    {
      x += 2; y += 1;
    }
    assert(50 < x && x <= 52);
  }
\end{lstlisting}

Si le domaine n'est pas celui des constantes, l'analyse en avant est
capable de garantir que l'assertion est toujours vérifiée.
Sans élargissement, l'analyse ne terminerait pas, à cause de
l'incrémentation de \verb|y| ; les itérations décroissantes permettent
ensuite de raffiner la valeur de \verb|x|.


\subsection{Comportement vis-à-vis des assertions}
\begin{lstlisting}[frame=single]
int a = rand(1, 100);
int b = rand(1, 100);
int main()
{
  assert(a > b);
  assert(a + b >= 3);
}
\end{lstlisting}

Si les assertions sont assumées, la première assertion échoue mais la
seconde assertion est toujours vérifiée. Au contraire si les
assertions sont ignorées, les deux assertions échouent.

\subsection{Analyse en arrière: trace d'échec}
\begin{lstlisting}[frame=single]
int x = rand(-100, 100);
int f(int alpha)
{
    if(x == alpha) { x = alpha + 1; }
    else           { x = alpha    ; }
}
int main()
{
    f(9);
    assert(x != 10);
}
\end{lstlisting}

L'analyse en arrière renvoie une trace d'échec du programme passant
par la première branche de la conditionnelle.

\lstinputlisting[frame=single, language=Bash]{examples/ex8.out}


\subsection{Tout ensemble}
\begin{lstlisting}[frame=single]
int x = rand(-100, 100);
int y = rand(-100, 100);
int z = rand(90, 300);
int f(int alpha, int beta)
{
    while(x >= alpha && x <= beta)
    {
	x++;
    }
}

int main()
{
    f(y, z);
    assert(x < y || x > z);
}
\end{lstlisting}
Les domaines relationnels permettent de vérifier que l'assertion est
toujours valide.

\end{document}